\documentclass[fleqn]{article}

%% Language and font encodings
\usepackage[english]{babel}
\usepackage[utf8x]{inputenc}
\usepackage[T1]{fontenc}

%% Sets page size and margins
\usepackage[a4paper,top=3cm,bottom=2cm,left=3cm,right=3cm,marginparwidth=1.75cm]{geometry}
\setlength{\mathindent}{1cm}
%% Useful packages
\usepackage[table,xcdraw]{xcolor}
\usepackage{amsmath}
\usepackage{graphicx}
\usepackage[colorinlistoftodos]{todonotes}
\usepackage[colorlinks=true, allcolors=blue]{hyperref}
\usepackage{listings}
\usepackage{color}
\usepackage{amsmath}

\definecolor{dkgreen}{rgb}{0,0.6,0}
\definecolor{gray}{rgb}{0.5,0.5,0.5}
\definecolor{mauve}{rgb}{0.58,0,0.82}

\title{Offentlige midler, offentlig kode}
\author{}


\begin{document}
\maketitle

\setcounter{secnumdepth}{0}

\section{Borgerforslag: Offentlige midler, offentlig kode}

\subsection{Meta}
Formatet for dette forslag er en henstilling til, at offentlig IT udvikles med åbne
standarder og med open-source modeller. Vi skal gøre det så konkret som muligt, men 
samtidig drage paralleller til relevante samfundsforhold, så hr. og fru Danmark også 
kan forstå konteksten.

Når en landevej går i stykker, kan vi reparere den. Men når et IT projekt går i stykker,
kuldsejler det. Og de milliardstore regninger sendes videre til os - borgerne.

I Danmark er IT en vigtig del af vores hverdag. Danmark ville ikke virke uden IT.
Problemet med den IT vi har i dag er, at vi ikke kan styre den. Når staten køber
et IT system i dag køber de katten i sækken: vi ved simpelthen ikke hvordan de
er bygget. 

Problemet er enormt fordi vi lægger vores liv i hænderne på IT systemerne. Og de
ved *alt* om os. Millioner af sygdomsforløb, straffeattester og andre fortrolige
dokumenter bliver dagligt sendt igennem utallige IT platforme. Langt hovedparten
af dem er udenfor vores kontrol. Tilgengæld ved vi med sikkerhed at Amerikansk
og Britiske efterretningstjenester lytter med i vores digitale - og private - liv.

Det er statens opgave at sørge for kvalitet. Og det har de ikke gjort.
Løsningen er heldigvis simpel: vi skal have mere åbenhed. 
Ligesom man kan kigge ind i vores vejnet og vores kommunalplaner skal man kunne
kigge ind i den offentlige software. Softwaren skal være *open-source*.
Lige nu ved vi ikke hvordan meget af vores IT infrastruktur ser ud.
Den er ejet af de virksomheder som har vundet de offentlige udbud, ikke af os.
Fremover skal staten kræve at offentlig IT bliver *open-source*. Vi vil selv kunne
reparere vores systemer når det går i stykker, uden at det koster milliarder af
kroner. Og det manglede da bare: det er jo vores skattekroner.

Danmark er kommet godt fra start i den digitale revolution, men hvis vi fortsætter
med at bygge vores offentlige infrastruktur op om store udenlandske selskaber
vil Danmark sakke bagud.
Med dette forslag vil vi tvinge magthaverne til at starte forfra. Vi vil
skabe en sikker og nem digital hverdag, hvor vi, Danmark, har kontrollen.
Og som vi selv ejer. Hjælp os ved at skrive under. 
Hjælp os med at undgå endnu en Polsag, som kostede os allesammen flere
milliarder kroner.

Tak.

\subsection{Stikord}
Borgere skal have adgang til koden
Ingen vendor lock-in
Åbne standarder
Fokus på nye udbud og kontrakter; gamle kontrakter skal ikke ændres
Danmark skal være et forgangsland

\begin{thebibliography}{1}

\bibitem{Lauesen}, Lauesen, Soren: Working paper: Damages and damage causes in
large IT government projects, IT-University of Copenhagen, 2017.
\end{thebibliography}

\end{document}

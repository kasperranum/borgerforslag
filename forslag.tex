\documentclass[fleqn]{article}

%% Language and font encodings
\usepackage[english]{babel}
\usepackage[utf8x]{inputenc}
\usepackage[T1]{fontenc}

%% Sets page size and margins
\usepackage[a4paper,top=3cm,bottom=2cm,left=3cm,right=3cm,marginparwidth=1.75cm]{geometry}
\setlength{\mathindent}{1cm}
%% Useful packages
\usepackage[table,xcdraw]{xcolor}
\usepackage{amsmath}
\usepackage{graphicx}
\usepackage[colorinlistoftodos]{todonotes}
\usepackage[colorlinks=true, allcolors=blue]{hyperref}
\usepackage{listings}
\usepackage{color}
\usepackage{amsmath}

\definecolor{dkgreen}{rgb}{0,0.6,0}
\definecolor{gray}{rgb}{0.5,0.5,0.5}
\definecolor{mauve}{rgb}{0.58,0,0.82}

\title{
    Borgerforslag\\
    Offentlige midler, offentlig kode
}
\author{}


\begin{document}
\maketitle

\setcounter{secnumdepth}{0}

\subsection{Meta}

Formatet for dette forslag skal underbygge henstillingen til, at offentlig IT så vidt muligt udvikles
med åbne standarder og med open source modeller. Vi skal gøre det så konkret som muligt,
men samtidig drage paralleller til relevante samfundsforhold, så Hr. og Fru Danmark også 
kan forstå konteksten.

\subsection{Forslag}

\begin{large}
Folketinget anmodes om at stadfæste ved lov at al offentlig IT så vidt muligt skal licenseres
og udgives som \textit{open source}; og, i tilfælde hvor systemet ikke kan offentliggøres åbent, sikre staten ejerskab af infrastruktur og kode.
Koden bag de offentlige IT-systemer skal med andre ord være mere tilgængelige for både borger og stat, end de er i dag.
Det vil sikre billigere systemer, højere kvalitet, større sikkerhed og vækst.
\end{large}

\subsection{Introduktion}

Offentlige IT-projekter udgør en væsentlig del af den infrastruktur, der får Danmark til at fungere.
Når IT-systemer ikke fungerer, er det i sidste ende borgerne, der betaler---gennem skattekroner til løsninger, eller ved direkte at blive berørt af systemer, der ikke fungerer.
Den nuværende model for udvikling af IT-systemer i den offentlige sektor har gang på gang resulteret i projekter, der er gået over budget, eller har slået fuldstændig fejl, og aldrig er blevet taget i brug.\textsuperscript{[citation needed]}
Mange af disse problemer kunne være undgået \cite{ITU}.
Én tilgang til at sørge for at dårlige systemer ikke koster os så dyrt, ville være at ændre hvem, der som udgangspunkt ejer koden bag, der kører systemerne.

Den eneste rigtige\textsuperscript{[citation needed]}\footnote{Note fra Niels: Det jeg mener med [citation needed] er: her bør vi være forsigtige, og underbygge \emph{hvorfor} det er den eneste rigtige løsning.} løsning er at sikre åbenhed og rimelighed i vores livsvigtige digitale infrastruktur gennem åbne standarder og \textit{open source}\footnote{Note fra Niels: Jeg mener vi bør introducere termet open source bedre, og så lade være med at have det i kursiv efter første introduktion.} licenser.\footnote{Note fra Niels: jeg ville foretrække at fokusere på "ejerskab af kode" over "åbne standarder"... Jeg tror vi bliver for fanget i vores egen tech verden med "åbne standarder".}
Lige nu er store dele af vores digitale liv ejet af private virksomheder, som vi er afhængige af, idet de udbygger og reparerer vores offentlige IT.
Med dette forslag sikrer vi en fremtid hvor Danmark går forrest med moderne og sikker IT.

\subsection{Øget ejerskab giver billigere IT}

Når staten køber et IT system i dag køber de katten i sækken. Staten ejer ikke engang
systemet. Det gør den virksomhed, der har bygget det. Tænk hvis det også gjaldt for
Storebæltsbroen. Så ville vi være tvunget til at gå til det firma der byggede
broen i 1998 for at få lavet reparationer på den. Firmaet ville have monopol på alt
der har med broen at gøre, og det ville være dyrt.
Desværre er det præcis det, der sker med vores offentlige IT. Alene i årene 
2000-2014 har staten Danmark kastet 2,902 milliarder kroner væk på skrottet IT
\cite{dr8}. Det svarer til ca. 7.000 fuldtidsansatte folkeskolelærer i de 14 år.

Hvis systemerne havde været ejet af staten og udgivet som \textit{open source} kunne
vi---ligesom med IC4 togene---havde reddet og genbrugt store dele af dem, selv efter
projektet i første omgang gik galt.
Det er en langt bedre løsning på en bedre digital hverdag i Danmark. Vi skal kræve
at vores leverandører lever op til åbne standarder.\footnote{Note fra Niels: Her
synes jeg vi bør uddybe om ejerskab. For det er ikke bare åbne standarder, men
at staten ejer koden, der giver denne effekt. De andre punkter kommer mere konkret
fra åbne standarder.}

\subsection{Åbne standarder giver kvalitet}

Tag Danmarks love som et eksempel: de sørger for at staten og borgerne opfører sig 
ordentligt og de skaber tryghed og livskvalitet. Fordi vores IT systemer er lukkede
kan vi ikke se om de overholder vores love. Vi kan ikke se om det er kvalitet eller skrammel.

Uafhængig forskning har gentagne gange vist at \text{open source}-systemer ikke bare er billigere,
men skaber tryghed samtidig med at de levere systemerne hurtigere og i en højere kvalitet \cite{Samoladas, Reynolds}.
Forklaringen ligger i at åben kode bliver bygget og set af mange flere mennesker. Og fire
øjne er som bekendt bedre end to.

\subsection{Åbne standarder giver sikkerhed}

Millioner af sygdomsforløb, straffeattester og andre fortrolige
dokumenter bliver dagligt sendt igennem vores offentlige IT platforme. Men langt størstedelen
af vores IT infrastruktur er udenfor vores kontrol.
Et IT system der er \textit{open source} betyder ikke at hele systemet er åbent. 
Hvis vi sammenligner med fx et kloaksystem, ønsker vi kun at gøre tegningerne
tilgængelige. Selve kloakken - og dens indhold - får lov til at være i fred.
Det betyder faktisk i sidste ende at IT systemerne bliver mere sikre \cite{Samoladas, Reynolds}.

\subsection{Bredere adgang til infrastrukturen leder til innovation og vækst}

Danmark er kommet godt fra start i den digitale revolution, men hvis vi fortsætter
med at udlicitere vores digitale infrastruktur, vil vi tabe vores digitale pust.
Et problem vi har set igen og igen er, at vores systemer ikke kan tale sammen
\cite{ITU, Lauesen}. Inderhavnsbroen i København er et eksempel fra den
fysiske verden, hvor de to sider ikke kunne samles på midten.

De knap 3 milliarder vi har smidt væk imellem 2000 og 2014 har ikke hjulpet Danmark til
at komme foran i den digitale udvikling. Tværtimod. Men det kan de komme til fremover, hvis vi
insisterer på at vores digitale broer skal kunne forbindes og arbejde sammen.
Mere åbenhed i vores offentlige IT infrastruktur betyder ikke kun at vi får øget
kvalitet og mere sikkerhed. Det betyder at vi kan skabe bedre systemer.

Samtidig åbner offentlig adgang til statens IT-projekter op for innovation fra civilbefolkningen, der kan hjælpe med at udbygge eller bygge ovenpå statens systemer.
Det kan potentielt føre til en helt ny sektor af mindre virksomheder, der reelt laver IT til den offentlige sektor, men med mindre nicher af befolkningen som deres målgruppe.

Danmark vil som resultat få ny infrastruktur, der understøtter opgaver der ikke før var politisk vilje eller budget til at understøtte; og samtidig vil de nye muligheder for opgaver skabe arbejdspladser.

\vskip10pt
Med dette forslag vil vi bane vejen for en sikker og nem digital
hverdag, hvor vi, Danmark og danskerne, har kontrollen, og hvor vi ejer de ting vi køber.
Det manglede i øvrigt bare: det er vores skattekroner.
\vskip5pt
Hjælp os ved at skrive under.

\vskip5pt
Tak.

\subsection{Stikord}
Borgere skal have adgang til koden
Ingen vendor lock-in
Åbne standarder
Fokus på nye udbud og kontrakter; gamle kontrakter skal ikke ændres
Danmark skal være et forgangsland

\begin{thebibliography}{1}

\bibitem{Lauesen} Søren Lauesen: Working paper: Damages and damage causes in
large IT government projects, IT-University of Copenhagen, 2017. Se
\url{https://itu.dk/~slauesen/Papers/DamageCaseStories_Latest.pdf}.

\bibitem{dr8} Skaaning, Jakob: Her er 8 store offentlige it-skandaler til milliarder, DR
2014. Se
\url{https://www.dr.dk/nyheder/penge/her-er-8-store-offentlige-it-skandaler-til-milliarder}.

\bibitem{ITU} Vibeke Arildsen og Søren Lauesen: Offentlige it-skandaler kan forhindres, ITU 2017.
Se \url{https://www.itu.dk/om-itu/presse/nyheder/2017/offentlige-it-skandaler-kan-forhindres}.

\bibitem{Reynolds} Reynolds CJ, Wyatt JC: Open Source, Open Standards, and Health Care Information
Systems, J Med Internet Res 2011;13(1):e24. Se \url{http://www.jmir.org/2011/1/e24/}.

\bibitem{Samoladas} Ioannis Samoladas og Ioannis Stamelos: Assessing Free/Open Source Software
Quality 2018. Se
\url{https://www.researchgate.net/publication/228715685_Assessing_freeopen_source_software_quality}.

\end{thebibliography}

\end{document}

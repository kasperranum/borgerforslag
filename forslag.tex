\documentclass[fleqn]{article}

%% Language and font encodings
\usepackage[english]{babel}
\usepackage[utf8x]{inputenc}
\usepackage[T1]{fontenc}

%% Sets page size and margins
\usepackage[a4paper,top=3cm,bottom=2cm,left=3cm,right=3cm,marginparwidth=1.75cm]{geometry}
\setlength{\mathindent}{1cm}
%% Useful packages
\usepackage[table,xcdraw]{xcolor}
\usepackage{amsmath}
\usepackage{graphicx}
\usepackage[colorinlistoftodos]{todonotes}
\usepackage[colorlinks=true, allcolors=blue]{hyperref}
\usepackage{listings}
\usepackage{color}
\usepackage{amsmath}

\definecolor{dkgreen}{rgb}{0,0.6,0}
\definecolor{gray}{rgb}{0.5,0.5,0.5}
\definecolor{mauve}{rgb}{0.58,0,0.82}

\title{Offentlige midler, offentlig kode}
\author{}


\begin{document}
\maketitle

\setcounter{secnumdepth}{0}

\section{Borgerforslag: Offentlige midler, offentlig kode}

\subsection{Meta}
Formatet for dette forslag skal underbygge henstillingen til, at offentlig IT udvikles
med åbne standarder og med open source modeller. Vi skal gøre det så konkret som muligt,
men samtidig drage paralleller til relevante samfundsforhold, så hr. og fru Danmark også 
kan forstå konteksten.

\subsection{Forslag}
\begin{large}
Folketinget anmodes om at stadfæste ved lov at al offentlig IT fremover skal licenseres
og udgives som \textit{open source}. Koden bag de offentlige IT systemer skal med andre ord
være tilgængelig for både borger og stat. Det vil sikre billigere systemer, højere
kvalitet, sikkerhed og vækst.
\end{large}

\subsection{Introduktion}
Offentlige IT projekter vedrører os alle, og de mange kuldsejlede IT systemer har
kostet urimeligt dyrt, både økonomisk og i tabt arbejdsfortjeneste. Problemet er enormt
fordi det sker igen og igen, og fordi det sagtens kunne have været undgået \cite{ITU}.
Imellemtiden ville vi have været bedre tjent uden de forældede og usikre systemer.

Den eneste rigtige løsning er at sikre åbenhed og rimelighed i vores livsvigtige
digitale infrastruktur gennem åbne standarder og \textit{open source} licenser.
Lige nu er store dele af vores digitale liv ejet af private virksomheder, som vi
er afhængige af til at udbygge og reparere vores offentlige IT. Med vores
forslag sikrer vi en lys fremtid hvor Danmark går forrest med moderne og sikker IT.
% Er der kilder nok? Skal det udbygges?

\subsection{Billigere IT}
Når staten køber et IT system i dag køber de katten i sækken. Staten ejer ikke engang
systemet. Det gør den virksomhed, der har bygget det. Tænk hvis det også gjaldt for
Storebæltsbroen. Så ville vi være tvunget til at gå til det firma der byggede
broen i 1998 for at reparere den. Det kaldes monopolvirksomhed, og det er dyrt.
Men desværre er det præcis det der sker med vores offentlige IT. Alene i årene 
2000-2014 har staten Danmark kastet 2,902 milliarder kroner væk på skrottet IT
\cite{dr8}. Det svarer til ca. 7'000 fuldtidsansatte folkeskolelærer i de 14 år.

Hvis systemerne havde været ejet af staten og udgivet som \textit{open source} kunne vi
- ligesom med IC4 togene - havde reddet og genbrugt store dele af dem. 
Det er den eneste rigtige løsning på en bedre digital hverdag i Danmark. Vi skal kræve
at vores leverandører lever op til åbne standarder. 

\subsection{Kvalitet}
Tag Danmarks love som et eksempel: de sørger for at staten og borgerne opfører sig 
ordentligt og de skaber tryghed og livskvalitet. Fordi vores IT systemer er lukkede
kan vi ikke se om de overholder vores love. Vi kan ikke se om det er kvalitet eller skrammel.

Uafhængig forskning har gentagne gange vist at \text{open source} ikke bare er billigere,
men skaber tryghed samtidig med at de levere systemerne hurtigere og i en højere kvalitet \cite{Samoladas, Reynolds}.
Forklaringen ligger i at åben kode bliver bygget og set af mange flere mennesker. Og fire
øjne er som bekendt bedre end to.

\subsection{Sikkerhed}
Millioner af sygdomsforløb, straffeattester og andre fortrolige
dokumenter bliver dagligt sendt igennem vores offentlige IT platforme. Men langt størstedelen
af vores IT infrastruktur er udenfor vores kontrol.
Et IT system der er \textit{open source} betyder ikke at hele systemet er åbent. 
Hvis vi sammenligner med fx et kloaksystem, ønsker vi kun at gøre tegningerne
tilgængelige. Selve kloakken - og dens indhold - får lov til at være i fred.
Det betyder faktisk i sidste ende at IT systemerne bliver mere sikre \cite{Samoladas, Reynolds}.

\subsection{Vækst}
Danmark er kommet godt fra start i den digitale revolution, men hvis vi fortsætter
med at udlicitere vores digitale infrastruktur, vil vi tabe vores digitale pust.
Et problem vi har set igen og igen er, at vores systemer ikke kan tale sammen
\cite{ITU, Lauesen}. Inderhavnsbroen i København er et eksempel fra den
fysiske verden, hvor de to sider ikke kunne samles på midten.

De knap 3 milliarder vi har smidt væk imellem 2000 og 2014 har ikke hjulpet Danmark til
at komme foran i den digitale udvikling. Tværtimod. Men det kan de komme til fremover, hvis vi
insisterer på at vores digitale broer skal kunne forbindes og arbejde sammen.
Mere åbenhed i vores offentlige IT infrastruktur betyder ikke kun at vi får øget
kvalitet og mere sikkerhed. Det betyder at vi kan skabe bedre systemer. Og det giver os tid
til at bekymre os om det der er vigtigt: flere jobs og flere penge til os alle i Danmark.

\vskip10pt
Med dette forslag vil vi bane vejen for en sikker og nem digital
hverdag, hvor vi, Danmark og danskerne, har kontrollen, og hvor vi ejer de ting vi køber.
Det manglede i øvrigt bare: det er vores skattekroner.
\vskip5pt
Hjælp os ved at skrive under.

\vskip5pt
Tak.

\subsection{Stikord}
Borgere skal have adgang til koden
Ingen vendor lock-in
Åbne standarder
Fokus på nye udbud og kontrakter; gamle kontrakter skal ikke ændres
Danmark skal være et forgangsland

\begin{thebibliography}{1}

\bibitem{Lauesen} Søren Lauesen: Working paper: Damages and damage causes in
large IT government projects, IT-University of Copenhagen, 2017. Se
\url{https://itu.dk/~slauesen/Papers/DamageCaseStories_Latest.pdf}.

\bibitem{dr8} Skaaning, Jakob: Her er 8 store offentlige it-skandaler til milliarder, DR
2014. Se
\url{https://www.dr.dk/nyheder/penge/her-er-8-store-offentlige-it-skandaler-til-milliarder}.

\bibitem{ITU} Vibeke Arildsen og Søren Lauesen: Offentlige it-skandaler kan forhindres, ITU 2017.
Se \url{https://www.itu.dk/om-itu/presse/nyheder/2017/offentlige-it-skandaler-kan-forhindres}.

\bibitem{Reynolds} Reynolds CJ, Wyatt JC: Open Source, Open Standards, and Health Care Information
Systems, J Med Internet Res 2011;13(1):e24. Se \url{http://www.jmir.org/2011/1/e24/}.

\bibitem{Samoladas} Ioannis Samoladas og Ioannis Stamelos: Assessing Free/Open Source Software
Quality 2018. Se
\url{https://www.researchgate.net/publication/228715685_Assessing_freeopen_source_software_quality}.

\end{thebibliography}

\end{document}

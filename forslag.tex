\documentclass[fleqn]{article}

%% Language and font encodings
\usepackage[english]{babel}
\usepackage[utf8x]{inputenc}
\usepackage[T1]{fontenc}

%% Sets page size and margins
\usepackage[a4paper,top=3cm,bottom=2cm,left=3cm,right=3cm,marginparwidth=1.75cm]{geometry}
\setlength{\mathindent}{1cm}
%% Useful packages
\usepackage[table,xcdraw]{xcolor}
\usepackage{amsmath}
\usepackage{graphicx}
\usepackage[colorinlistoftodos]{todonotes}
\usepackage[colorlinks=true, allcolors=blue]{hyperref}
\usepackage{listings}
\usepackage{color}
\usepackage{amsmath}

\definecolor{dkgreen}{rgb}{0,0.6,0}
\definecolor{gray}{rgb}{0.5,0.5,0.5}
\definecolor{mauve}{rgb}{0.58,0,0.82}

\title{Offentlige midler, offentlig kode}
\author{}


\begin{document}
\maketitle

\setcounter{secnumdepth}{0}

\section{Borgerforslag: Offentlige midler, offentlig kode}

\subsection{Meta}
Formatet for dette forslag er en henstilling til, at offentlig IT udvikles med åbne
standarder og med open source modeller. Vi skal gøre det så konkret som muligt, men 
samtidig drage paralleller til relevante samfundsforhold, så hr. og fru Danmark også 
kan forstå konteksten.

\subsection{Forslag}
\begin{large}
Folketinget anmodes om at stadfæste ved lov at al offentlig IT fremover skal licenseres
og udgives som *open-source*. Koden bag de offentlige IT systemer skal med andre ord
være tilgængelig for både borger og stat. Det vil sikre 1) kvalitet,
2) konkurrencedygtighed, 3) sikkerhed og 4) vækst.
\end{large}

\subsection{Introduktion}
Offentlige IT projekter vedrører os alle, og de mange kuldsejlede IT systemer har
været urimeligt dyre. Både økonomisk og i tabt arbejdsfortjeneste og konkurrenceevne.
Den eneste rigtige løsning er at sikre at vores offentlige investeringer forbliver
offentlige. Vi er lige nu dybt afhængige af dyre IT monopoler til at udbygge og
reparere vores livsvigtige IT infrastruktur.
% Jeg er ikke helt tilfreds med ovenstående... Hvordan kan vi indramme problemet
% mere præcist?

\subsection{Kvalitet}
Når staten køber et IT system i dag køber de katten i sækken: der er simpelthen ingen
der ved hvad de køber. Vi ejer ikke engang systemet. Det gør den virksomhed, der har bygget
det. Tænk hvis det også gjaldt for Storebæltsbroen. Så var vi tvunget til at gå til den
selvsamme virksomhed der byggede broen i 1998. Det kaldes monopolvirksomhed, og det er dyrt.
Men desværre er det præcis det der sker med vores offentlige IT.
Vi synes at vi skal eje de ting vi køber. Og vi vil gerne selv bestemme hvornår en vej skal
repareres eller bygges ud, tak.

Det manglede i øvrigt bare: det er jo vores skattekroner.
% Indsæt kilde!

\subsection{Konkurrencedygtighed}
Danmark er kommet godt fra start i den digitale revolution, men hvis vi fortsætter
med at sælge vores offentlige IT infrastruktur til private monopoler, vil vi hurtigt sakke bagud.
Det største problem er, at vores systemer ikke arbejder sammen. De bliver bygget hver
for sig, uden at kunne samarbejde. Ligesom Inderhavnsbroen i København, der ikke kunne
samles på midten. % Hmm, hjælper denne sammenligning?
Hvis vi vil have en bedre digital hverdag, skal vi kræve at leverandørerne lever op til 
åbne standarder - som vi har bestemt.
Ellers vil vi se endnu flere kuldsejlede projekter og i sidste ende miste vores
konkurrenceevne indenfor IT.

\subsection{Sikkerhed}
Millioner af sygdomsforløb, straffeattester og andre fortrolige
dokumenter bliver dagligt sendt igennem utallige IT platforme. Men langt størstedelen
af vores IT infrastruktur er udenfor vores kontrol.
Et IT system der er *open-source* betyder ikke at hele systemet er åbent. 
Hvis vi sammenligner Polsag med Koldings kloakker fx, ønsker vi kun at gøre tegningerne
tilgængelige. Selve kloakken - og dets indhold - får lov til at være i fred.
Det betyder faktisk i sidste ende at IT systemerne bliver mere sikre. Forklaringen ligger
i at åben kode bliver bygget og set af mange flere mennesker. Og fire øjne er som bekendt
bedre end to. Tro os, det har været en kendsgerning inden for IT i meget lang tid.
% Indsæt kilde

\subsection{Vækst}
IT systemerne bliver vigtigere og vigtigere for at Danmark kan hænge sammen. Og fremfor
at hjælpe os ind i det 21. århundrede skaber de store problemer for os. De holder os faktisk
tilbage teknologisk. Alligevel smider vi milliarder af skattekroner væk på systemer, der
ikke virker.
Det er statens opgave at sørge for kvalitet, og det har de ikke gjort \cite{Lauesen}.
Mere åbenhed i vores offentlige IT infrastruktur betyder ikke kun at vi får øget
kvalitet og mere sikkerhed. Det betyder at vi kan blive bedre til at arbejde sammen og
skabe vækst. Det giver flere jobs og giver flere penge til os alle.

\vskip10pt
Med dette forslag vil vi bane vejen for en sikker og nem digital
hverdag, hvor vi, Danmark og danskerne, har kontrollen, og hvor vi ejer de ting vi køber.
Hjælp os ved at skrive under.

\vskip10pt
Tak.

\subsection{Stikord}
Borgere skal have adgang til koden
Ingen vendor lock-in
Åbne standarder
Fokus på nye udbud og kontrakter; gamle kontrakter skal ikke ændres
Danmark skal være et forgangsland

\begin{thebibliography}{1}

\bibitem{Lauesen} Lauesen, Søren: Working paper: Damages and damage causes in
large IT government projects, IT-University of Copenhagen, 2017. Se
\url{https://itu.dk/~slauesen/Papers/DamageCaseStories_Latest.pdf}.
\end{thebibliography}

\end{document}

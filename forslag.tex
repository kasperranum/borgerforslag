\documentclass[fleqn]{article}

%% Language and font encodings
\usepackage[english]{babel}
\usepackage[utf8x]{inputenc}
\usepackage[T1]{fontenc}

%% Sets page size and margins
\usepackage[a4paper,top=3cm,bottom=2cm,left=3cm,right=3cm,marginparwidth=1.75cm]{geometry}
\setlength{\mathindent}{1cm}
%% Useful packages
\usepackage[table,xcdraw]{xcolor}
\usepackage{amsmath}
\usepackage{graphicx}
\usepackage[colorinlistoftodos]{todonotes}
\usepackage[colorlinks=true, allcolors=blue]{hyperref}
\usepackage{listings}
\usepackage{color}
\usepackage{amsmath}

\definecolor{dkgreen}{rgb}{0,0.6,0}
\definecolor{gray}{rgb}{0.5,0.5,0.5}
\definecolor{mauve}{rgb}{0.58,0,0.82}

\title{Offentlige midler, offentlig kode}
\author{}


\begin{document}
\maketitle

\setcounter{secnumdepth}{0}

\section{Borgerforslag: Offentlige midler, offentlig kode}

\subsection{Meta}
Formatet for dette forslag er en henstilling til, at offentlig IT udvikles med åbne
standarder og med open-source modeller. Vi skal gøre det så konkret som muligt, men 
samtidig drage paralleller til relevante samfundsforhold, så hr. og fru Danmark også 
kan forstå konteksten.

\subsection{Forslag}
Når en landevej går i stykker, kan staten reparere den. Når et IT system går ned er
vi som land magtesløse. Vi er dybt afhængige af dyre IT monopoler til at reparere de
livsvigtige IT systemer, og det sker alt for tit at hele projekter kuldsejler.
Tilbage står staten med milliardstore regninger, der sendes videre til os - borgerne.
% Indsæt kilde!

I Danmark er IT en livsvigtig del af vores hverdag og vores infrastruktur. Men når
staten køber et IT system i dag køber vi katten i sækken: vi ved simpelthen ikke
hvad vi køber, fordi vi ikke må se hvordan systemet fungerer. Vi ejer det ikke engang.
Det gør den virksomhed, der har bygget det. Tænk hvis det også galdt for storebæltsbroen.

Problemet er enormt fordi vi smider milliarder af kroner væk på systemer, der ikke virker.
Det er statens opgave at sørge for kvalitet. Og det har de ikke gjort \cite{Lauesen}.
Løsningen er heldigvis simpel: vi skal have mere åbenhed.
Ligesom man kan kigge ind i vores vejnet og vores kommunalplaner skal man kunne
kigge ind i IT systemerne. Softwaren skal være offentligt tilgængelig, eller
*open-source*.

Millioner af sygdomsforløb, straffeattester og andre fortrolige
dokumenter bliver dagligt sendt igennem utallige IT platforme. Men langt størstedelen
af vores IT infrastruktur er ejet af de virksomheder, og ikke af os borgere.
Fremover skal staten kræve at nyudviklet IT bliver solgt til os som *open-source*.
Vi vil kunne se hvordan det virker og kunne reparere vores systemer når de går i stykker
- uden at betale milliarder af kroner.
Det manglede i øvrigt bare: det er jo vores skattekroner.

Danmark er kommet godt fra start i den digitale revolution, men hvis vi fortsætter
med at sælge vores offentlige IT infrastruktur til udenlandske monopoler vil vi
hurtigt sakke bagud.
Med dette forslag vil vi rette op og starte forfra. Vi vil en sikker og nem digital
hverdag, hvor vi, Danmark, har kontrollen. Og hvor vi ejer de ting vi køber.
Hjælp os ved at skrive under.

Tak.

\subsection{Stikord}
Borgere skal have adgang til koden
Ingen vendor lock-in
Åbne standarder
Fokus på nye udbud og kontrakter; gamle kontrakter skal ikke ændres
Danmark skal være et forgangsland

\begin{thebibliography}{1}

\bibitem{Lauesen} Lauesen, Søren: Working paper: Damages and damage causes in
large IT government projects, IT-University of Copenhagen, 2017.
\end{thebibliography}

\end{document}
